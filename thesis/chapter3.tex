\chapter{Conclusions}
\label{conclusionchap}

The new codebase for the 2018 University of Connecticut ECE Senior Design project titled ``Smartwatch Device and App for Continuous Glucose Monitoring'' successfully realizes the firmware for a custom smartwatch that measures and graphs glucose data as well as performs tasks typically expected of a smartwatch, such as wirelessly communicating with and displaying notifications from a connected Bluetooth device, providing the current time and date, entering a sleep mode during inactivity, and displaying the wireless connection status and current battery level. The resulting codebase is well-structured, as it is separated into a number of logical components with descriptive file names that are organized into subdirectories of a single directory, each component has an interfacing core that provides a single entry point to the component, and each top-level function is written to be as brief and self-descriptive as possible. The codebase is also easily adaptable, as a number of different displays can be implemented without changing the core of the display code, extending the Bluetooth command format is simple, and the measurement capabilities of the smartwatch can be adapted for tasks other than reading glucose.

Future improvements that can be made to the codebase include entering deep sleep mode when the battery is at a critical level, providing display driver options for using different color resolutions and larger screens, implementing priority in the Bluetooth messaging format, encrypting stored data to protect personal user information, and employing a real-time operating system to run the tasks of the smartwatch concurrently.
